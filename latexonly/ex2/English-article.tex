\documentclass[UTF8]{ctexart}

\usepackage{tipa}

\title{English article}
\author{梁彦泽}
\date{\today}

\newcommand\Emph{\textbf}

\begin{document}


\maketitle
\tableofcontents
\newpage

\section{text}
Researchers who are \Emph{unfamiliar} \footnote{a. 不熟悉的} with the culture and \Emph{ethnic} groups they are studying
must take extra \Emph{precautions} to \Emph{shed} any \Emph{biases} they bring with them from their own culture. For example,
they must make sure they \Emph{construct measures} that meaningful for each of the cultural or ethnic \Emph{minority} groups being studied.


In conducting research on culture and ethnic minority issues, investigator distinguish between the emic approach and the 
etic approach. In the emic approach, the goal is to describe behavior in one culture or ethnic group in term that are meaningful
and important to the people in that culture or ethnic group, without regard to other cultures or ethnic groups. In the etic approach, the goal is to
describe behavior so that generalizations can be made across cultures. If researchers construct a questionnaire in an emic fashion, their concern is only that the 
questions are meaningful to the particular culture or ethnic group being studied. If however, the researchers construct a questionnaire in an etic fashion , they
want to include questions that reflect concepts familiar to all cultures involved.

How might the emic and etic approaches be reflected int the study of family processes? In the emic approach, the researchers might choose to focus only on middle-class
White families, without regard for whether the information obtained in the study can be generalized or is appropriate for ethmic minority groups. In a subquent study,
the researchers may decide to adopt an etic approach by studying not only middle-class White families, but also lower-income White families, Black American families,
Spanish American families, and Asian American families. In studying ethnic minority families, the researchers would likely discover that the extended family is more 
frequently a support system in ethnic minority families than in White American families. If so, the emic approach would reval a different pattern of family interaction
than, would the etic approach, documenting that research with middle-class White families cannot always be generalized to all ethnic groups.
\end{document}
