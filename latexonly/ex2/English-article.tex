\documentclass[UTF8]{ctexart}

\usepackage{tipa}

\title{English article}
\author{梁彦泽}
\date{\today}

\newcommand\Emph{\textbf}

\begin{document}


\maketitle
\tableofcontents
\newpage

\section{text 1}
Researchers who are \Emph{unfamiliar} \footnote{a. 不熟悉的} with the culture and \Emph{ethnic} groups they are studying
must take extra \Emph{precautions} to \Emph{shed} any \Emph{biases} they bring with them from their own culture. For example,
they must make sure they \Emph{construct measures} that meaningful for each of the cultural or ethnic \Emph{minority} groups being studied.


In conducting research on culture and ethnic minority issues, investigator distinguish between the emic approach and the 
etic approach. In the emic approach, the goal is to describe behavior in one culture or ethnic group in term that are meaningful
and important to the people in that culture or ethnic group, without regard to other cultures or ethnic groups. In the etic approach, the goal is to
describe behavior so that generalizations can be made across cultures. If researchers construct a questionnaire in an emic fashion, their concern is only that the 
questions are meaningful to the particular culture or ethnic group being studied. If however, the researchers construct a questionnaire in an etic fashion , they
want to include questions that reflect concepts familiar to all cultures involved.

How might the emic and etic approaches be reflected int the study of family processes? In the emic approach, the researchers might choose to focus only on middle-class
White families, without regard for whether the information obtained in the study can be generalized or is appropriate for ethmic minority groups. In a subsequent study,
the researchers may decide to adopt an etic approach by studying not only middle-class White families, but also lower-income White families, Black American families,
Spanish American families, and Asian American families. In studying ethnic minority families, the researchers would likely discover that the extended family is more 
frequently a support system in ethnic minority families than in White American families. If so, the emic approach would reval a different pattern of family interaction
than, would the etic approach, documenting that research with middle-class White families cannot always be generalized to all ethnic groups.

\section{text 2}

War may be a natural expression of biological instincts and drives toward aggression in the human species. Natural impulses
of anger, hostility, and territoriality are expressed through acts of violence. There are all qualities that humans share
with animals. Aggression is a kind of innate survival mechanism, an instinct for self-preservation, that allows animals to defend themselves from threats to their existence.
But, on the other hand, human violence shows evidence of being a learned behavior, In the case of human aggression, violence cannot be simple reduced to an instinct. The
many expressions of human societies violence has a social function: It is a strategy for creating of destroying forms of social order. Religious traditions have taken a
leading role in directing the powers of violence. We wil look at the ritual and ethical patterns within which human violence has been directed.

The violence within a society is controlled through institutions of law. The more developed a legal system becomes, the more society takes responsibility for the discovery, control, and punishment of violent acts. In most tribal societies the only means to deal with an act of violence is revenge. Each family group may have the resopnsibility
for personally carrying out judgment and punishment upon the person who committed the offense. But in legal systems, the responsibility for revenge becomes depersonalized
and diffused. The society assumes the responsibility for protecting individuals from violence. In cases where they cannot be protected, the society is responsibility
for imposing punishment. In a state controlled legal system, individuals are removed from the cycles of revenge motivated by acts of violence, and the state assumes
responsibility for their protection.

The other side of a state legal apparatus is a state military apparatus. While the one protects the individual from violence, the other sacrifices the individual to violence 
in the interests of the state, In war the state affirms its supreme power over the individual within its own borders. War is not simply a trial by combat to settle disputes
between states; it is the moment when the state makes its most powerful demands upon its people for their recommitment, allegiance, and supereme sacrifice. Times of war
test a community's deepest religious and ethical commitments.
 
\end{document}
