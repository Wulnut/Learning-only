\documentclass[UTF8, a4paper, 12pt]{ctexart}

\usepackage{lipsum}
%\usepackage{xeCJK} %% 中文字体扩展管理宏包,务必添加!!
\usepackage{amsthm}
\usepackage{newtxtext}
\usepackage[urlcolor=blue]{hyperref}
\usepackage{threeparttable}
\usepackage{setspace}
\usepackage{titlesec}
\usepackage{float}
\newcommand{\upcite}[1]{\textsuperscript{\textsuperscript{\cite{#1}}}}
\usepackage{fancyhdr}
\usepackage{booktabs}
\usepackage{listings}
\usepackage{clrscode}

\zihao{4}
\begin{document}
%%\youyuan

\begin{center}
\LARGE
    \textbf{\LaTeX{}学习练习} \\
    \vspace{0.2em}
    \large
    作者:梁彦泽 \hspace{1cm} 博客:www.wulnut.top \\
    \rule[0.5 \baselineskip]{\textwidth}{0.5pt}
\end{center}

%% \rule命令是用来画分割线的



\textbf{摘\ 要} \\
\large
微分几何是运用微积分的理论研究空间的几何性质的数学分支学科。
古典微分几何研究三维空间中的曲线和曲面,而现代微分几何开始研究更一般的空间——流形。


\textbf{关键词}: 仔细\quad 仔细\quad 仔细\quad \\

\begin{center}
\rule[0.1 \baselineskip]{\textwidth}{0.5pt}
\end{center}

\section{引言}
\indent
近代由于对高维空间的微分几何和对曲线、曲面整体性质的研究,使微分几何和拓扑学、变分学、李群理论等有了密切的关系,这些数学领域和微分几何互相渗透,已成为现代数学的中心课题之一。
微分几何在力学和一些工程技术问题方面有广泛的应用,比如,在弹性薄壳结构方面,在机械的齿轮啮合理论应用方面,都充分应用了微分几何学的理论。


\section{第一部分}
\indent
在曲面上有两条重要概念,就是曲面上的距离和角。比如,在曲面上由一点到另一点的路径是无数的,但这两点间最短的路径只有一条,叫做从一点到另一点的测地线。在微分几何里,要讨论怎样判定曲面上一条曲线是这个曲面的一条测地线,还要讨论测地线的性质等。另外,讨论曲面在每一点的曲率也是微分几何的重要内容。


%% bottomrule、midrule、toprule使用时都需要先引入包\usepackage{booktabs}才能使用
\begin{table}[H]
	\caption{表格}
	\centering
	\begin{tabular}{cccccccc}
		\toprule[1.5pt]
		年份  & 2006&2007&2008&2009&2010 & 2011 & 2012 \\
		\midrule
		A&57.95&58.187&59.1&59.652&60.22&61.072&61.418 \\
		B &55.7957 &58.3199&58.8548&59.9983&60.3769 &60.9841 &61.7716 \\		
		C&2.1543&0.1329&0.2452&0.3463&0.1569&0.0879&0.3536 \\		
		D&0.0372 &0.0023&0.0041&0.0058&0.0026&0.0014&0.0058 \\
		\bottomrule[1.5pt]
		\label{tab1}
	\end{tabular}
\end{table}


\begin{thebibliography}{9}%宽度9
    \zihao{-5}
    \bibitem{bib:one} 毕郑南.陈省身在数学领域的独特人生[J].兰台世界,2014,(4):68-69.
    \bibitem{bib:two} 王永青,刘海波,贾振元, 等.基于活动标架理论的加工目标曲面再设计及刀位计算[J].机械工程学报,2012,48(19):141-147.
\end{thebibliography}


\section*{matlab源代码}
\begin{lstlisting}[language=matlab]
    clc;clear;
    row = size(A)
    row = size(A,1)
    column = size(A,2)
    [row,column] = size(A)
\end{lstlisting}

\section*{C++}
\lstset{columns = flexible, numbers = left, numberstyle = \footnotesize, escapechar = `}
\begin{lstlisting}[language = C++]
    /*hello.cpp*/
    #include<iostream>
    int main() {
        cout << "hello" << endl; // `$\frac{1}{\sin x}$`
        return 0;
    }
\end{lstlisting}

\begin{codebox}
    \Procname{$\proc{Insertion-Sort}(A)$}
    \li \For $j \gets 2$ \To $\id{length}[A]$
    \li \Do
    $\id{key} \gets A[j]$
    \li \Comment Insert $A[j]$ into the sorted sequence
    $A[1 \twodots j-1]$.
    \li $i \gets j-1$
    \li \While $i > 0$ and $A[i] > \id{key}$
    \li \Do
    $A[i+1] \gets A[i]$
    \li $i \gets i-1$
    \End
    \li $A[i+1] \gets \id{key}$
    \End
\end{codebox}

\end{document}