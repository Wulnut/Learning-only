\documentclass[UTF8]{ctexart}

\title{杂谈沟谷定理}
\author{垃圾桶}
\date{\today}

\bibliography{plain}

\begin{document}

\maketitle
\tableofcontents
\section{勾股定理在古代}
自动生成一段文字,自动生成的文字, 自动生成一段文字,自动生成的文字,自动生成一段文字,自动生成的文字, 自动
\section{勾股定理的近代形式}
这一整个章节都是自动生成的文字都是自动生成的 文字都是 自动生成的文字
\begin{quote}
勾广三,股修四,经隅五
\end{quote}

\begin{quote}
\zihao{-5}\kaishu 引用的内容
\end{quote}

\begin{abstract}
这是一篇关于沟谷定理的小短文
\end{abstract}

\begin{equation}
    a ( b + c ) = ab + ac 
\end{equation}

$ a ( b + c) = ab + ac $

$\angle ABC = \pi / 2$

\begin{tabular}{|rrr|}
    \hline
    直角边 $a$ & 直角边 $b$ & 斜边 $c$ \\
    \hline

    3 &    4 &    5 \\
    5 &    12 &   13 \\

    \hline
\end{tabular}

中文测试
中文测试

中文测试
中文测试

中文测试

\bibliography{math}

\end{document}
